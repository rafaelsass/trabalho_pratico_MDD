\documentclass[12pt]{article}

\usepackage{sbc-template}
\usepackage{graphicx,url}
\usepackage[utf8]{inputenc}
\usepackage[brazil]{babel}
%\usepackage[latin1]{inputenc}  

     
\sloppy

\title{Proposta do TP1 - Mineração de Dados}

\author{Cecília Junqueira V. M. Pereira - 2022094888 \\ 
        Rafael Sebastião Arocho de Souza Silva - 2021035195}

\address{Universidade Federal de Minas Gerais (UFMG) \\
\email{ceciliajunq@ufmg.br} \email{rafaelsass@ufmg.br}}


\begin{document} 

\maketitle
     
\begin{resumo} 
    No Trabalho Prático 1 da matéria "Mineração de Dados", usaremos o conjunto de dados referentes às áreas verdes urbanas publicadas no Cadastro Ambiental Urbano (CAU).
\end{resumo}

\section{Contexto dos dados}
    A base de dados selecionada armazena informações acerca das áreas verdes urbanas presentes em diversos municípios do país. Há informações referentes ao tipo de área verde (chamada de "tipologia", que pode ser do tipo praça, parque, canteiro, jardim zoológico, agricultura urbana, áreas protegidas, entre outros), a data de registro da área e a cidade e estado onde ela se localiza, a área verde demarcada, somatório da área de lago/lagoa/represa encontrada dentro da área verde urbana cadastrada.

    Ao analisar os dados, podemos extrair as informações sobre quais cidades e estados são as que têm maior presença de áreas verdes e arborizadas, bem como de recursos relacionados a itens de segurança, infraestrutura, lazer, condições ambientais. Uma análise profunda dos dados nos permite entender quais investimentos nas áreas verdes urbanas são mais precários no país e em quais regiões. A base contém atributos acerca de se as áreas são conservadas ou não, se têm nascentes e flora nativas nos arredores e também aborda se há presença de infraestrutura para a ocupação da população na área.

\section{Sobre a tarefa a ser executada}
    Diante dos dados disponíveis na base, visamos realizar análises para descobrir quais estados brasileiros são os mais precários quanto ao investimento nas áreas urbanas, assim como descobrir qual o perfil mais comum das áreas verdes urbanas atualmente no país, tanto em relação à infraestrutura para uso pelos indivíduos, quanto à preservação da natureza.

    Tal tarefa é relevante, uma vez que essas áreas verdes urbanas são de grande importância aos municípios:
    
    - Ecológica: manutenção climática, preservação da biodiversidade, ajuda na infiltração da água das chuvas...
    
    - Econômica: valorização imobiliária, redução dos custos com a saúde pública (ambientes verdes incentivam a atividade física e reduzem o estresse, o que pode diminuir gastos com doenças crônicas e mentais)...
    
    - Social: bem-estar e qualidade de vida, espaços de convivência e lazer...
    
    Contudo, o Brasil enfrenta grandes desafios em manter tais áreas, devido à falta de verba e planejamento de manutenção, o que leva à degradação desses ambientes e ao desencadeamento de vários problemas urbanos e sociais. Por exemplo, se esses ambientes não forem bem cuidados, as vegetações que o compõe poderam acumular água parada e contriuir para a propagação da dengue, desencadeado uma situação de calamidade pública; deixarão as áreas verdes pouco adequadas para o uso humano, não satisfazendo o uso voltado ao lazer (que é um dos objetivos que essas áreas visam assegurar); e a falta de cuidado das áreas verdes não manteram um ambiente adequado para proteger a biodiversidade e a manutenção climática.

    Ao analisar os dados e capturar quais são os cidades com menos desenvolvimento desse setor e quais aspectos são menos desenvolvidos nas áreas verdes, as informações obtidas darão aos responsáveis por cuidar dessas áreas um significativo direcionamento sobre quais elementos compositores dessas áreas têm maior urgência de cuidado, para, então, impulsionar a melhora da presença das áreas verdes no país e usufruir os benefícios que a existências desses ambientes podem promover à vida urbana.

\section{Dados a serem utilizados}
    Para realizar as tarefas propostas, os dados disponíveis na base a serem utilizados serão:
    \begin{enumerate}
        \item Situação do registro
        \item Tipologia
        \item UF do registro
        \item Município do registro
        \item Média de avaliações
        \item Área verde demarcada
        \item Quantidade de nascentes
        \item Somatório de área a ser recuperada
        \item Área com vegetação perturbada, escassa ou ausente maior que 50\%
        \item Ausência de fauna nativa
        \item Espaço para eventos culturais e feiras
        \item Manutenção Periódica
        \item Acessibilidade para PcD
        \item Esporte e lazer (Outros)
        \item Mitigação de ilhas de calor
        \item Regulação da permeabilidade do solo
    \end{enumerate}

    
\clearpage
\bibliographystyle{sbc}
%\bibliography{referencias}

\end{document}
